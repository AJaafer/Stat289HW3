\documentclass[12pt]{article}
\usepackage{epsfig,amsmath,float}

%\leftmargin{0cm}
\setlength{\textheight}{8.5in}
%\def\home{/home/stroud}
\def\sty{\home/latex}
\def\bib{\home/bbl}
\def\picdir{/home/stroud/nonlin/pics}
\setlength{\textheight}{8.5in}
\setlength{\topmargin}{-0.5in}
\setlength{\oddsidemargin}{-.25truein}
\setlength{\evensidemargin}{0truein}
\setlength{\parindent}{0in}
\setlength{\parskip}{.3cm}
\renewcommand{\baselinestretch}{1.10}
\setlength{\textwidth}{6.93truein}
\leftskip 0truein

%%%%%% boldface characters %%%%%% 
\def\y{\mathbf{y}}
\newcommand{\be}{\mbox{\boldmath $\beta$}}
\newcommand{\Si}{\mbox{\boldmath $\Sigma$}}
\newcommand{\si}{\mbox{\boldmath $\sigma$}}
\newcommand{\bth}{\mbox{\boldmath $\theta$}}



\begin{document}

\begin{center}
{\large {\underline {\bf Stat 289 Homework 3 - Due Monday, April 12 (6:10 pm)}}}
\end{center}

\vspace{.2cm}
\noindent {\bf You are required to submit a hard copy document in class with answers to the following questions. This document must be generated using the LaTeX typesetting language.
It should also include any requested graphics for each question, integrated with
the text for that question. Do not just put all graphics at the end of the document!

\vspace{.2cm}
\noindent In addition to the hardcopy document, please also submit the LaTeX source code (.tex file) to me by email (stroud@gwu.edu). Please do all computational work using the 
software package R, and submit your R code to me by email as a separate plain text file.
\underline{Please do not consult with other students on this assignment.}}

\vspace{.2cm}
\noindent For Questions 1-7, you will need the SAT coaching dataset given in Table 1 below.
This dataset comes from a randomized experiment to test the effects of 8 different SAT 
coaching programs.  For each program $i$, an observed effect $y_i$ was measured as 
well as a standard error $\sigma_i$ of the treatment effect.  We want to analyze these 
treatments under a hierarchical model where
\begin{eqnarray*}
      y_i & \sim & \mbox{Normal}(\theta_i,\sigma^2_i)\\
 \theta_i & \sim & \mbox{Normal}(\mu,\tau^2)
\end{eqnarray*}
We will assume that $\si^2=(\sigma^2_1,\ldots,\sigma^2_n)$ are all known without error. 
We are interested in posterior inference for each $\theta_i$, which is the underlying true 
treatment effect for program $i$, as well as the common treatment mean $\mu$ and variance 
$\tau^2$.

\begin{enumerate}{\leftmargin=1em}
\item Verify that using a flat prior of $p(\mu,\tau)\propto 1$ corresponds to a prior for
$(\mu,\tau^2)$ of $p(\mu,\tau^2) \propto \tau^{-1}$.  With this prior, write out the 
full posterior distribution of the unknown parameters $\mu$, $\tau^2$ and 
$\bth=(\theta_1,\ldots,\theta_n)$.

\item The first approach is based on the realization that we can actually integrate out 
each $\theta_i$ from the above distribution, giving us the marginal distribution of our 
treatment effects $y_i$:
$$
y_i \sim \mbox{Normal}(\mu,\sigma^2_i+\tau^2)
$$
Write out the marginal posterior distribution of $p(\mu,\tau^2|\y,\si^2)$, and evaluate this posterior
distribution over a grid of values of $\mu$ and and $\tau^2$. Use a grid of (0,10) for $\tau^2$.

\item Use the grid sampling method to get 1000 samples from $p(\mu,\tau^2|\y,\si^2)$. 
Calculate the mean, median and 95\% posterior interval for $\mu$ and $\tau^2$.

\item Write out the distribution $p(\theta_i|\mu,\tau^2,\y,\si^2)$. Use the 1000 samples 
of $\mu$ and $\tau^2$ to draw 1000 samples of each $\theta_i$. Calculate the mean of each $\theta_i$ and 
compare to the observed treatment effects $y_i$.

\item For Questions 5-7, consider $\tau^2$ to be fixed and equal to $\tau^2 = \mbox{median}(\tau^2)$ 
that you calculated in Question 3. We now use a Gibbs sampler to draw samples from the posterior
distribution $p(\mu,\bth|\y,\si^2)$. Remember that $\tau^2$ is now a known quantity in this question, 
so we do not need to calculate the posterior distribution for it. Give the conditional distributions
of the remaining unknown parameters given the other parameters.

\item Use a Gibbs sampler based on the conditional distributions from the previous question
to obtain 1000 samples from $p(\mu,\bth|\y,\si^2)$. Make sure that you only use samples after
convergence and that your 1000 samples have low autocorrelation. Calculate the mean
and posterior interval for each $\theta_i$.

\item Use your samples of $\bth$ to calculate an estimate of the posterior probability that school
A offers the best program.

\vspace{.05cm}
Questions 8-11 are based on the bicycle data in Table 2 below. We will focus only on the 
first two rows of the table (the residential streets with bike routes).  We want to model 
the total amount of traffic $y_i$ on each street (eg. $y_1=74$) as follows:
%
\begin{eqnarray*}
        y_i & \sim & \mbox{Poisson}(\theta_i)\\
   \theta_i & \sim & \mbox{Gamma}(\alpha,\beta)
\end{eqnarray*}

\item Write out the posterior distribution of our unknown variables $p(\alpha,\beta,\bth|\y)$ using a flat prior distribution for $\alpha$ and $\beta$, $p(\alpha,\beta) \propto 1$.

\item We want to use a Gibbs sampler to get samples from $p(\alpha,\beta,\bth|\y)$. What is the conditional distribution of $\theta_i$ given all the other parameters?

\item The conditional distributions $p(\alpha,\beta|\y)$ and $p(\beta|\alpha,\y)$ are not 
easy to sample from, so you instead need to use a Metropolis (or Metropolis-Hastings) step 
to obtain samples from $p(\alpha,\beta|\bth,\y)$. Choose a proposal distribution for 
$\alpha$ and a proposal distribution for $\beta$.

\item Combine the algorithm from Question 10 with the conditional distribution from 
Question 9 to form a Gibbs sampler, which iterates between sampling:

1. sampling $\theta_i | \alpha,\beta,\y$ for each $i$.\\
2. sampling $\alpha,\beta|\bth,\y$.

Use this algorithm to obtain 1000 samples from $p(\alpha,\beta,\bth|\y)$. Make sure that 
you only use samples after convergence and that your 1000 samples have low autocorrelation.
Calculate means and 95\% posterior intervals for $\alpha$ and $\beta$.

\item Give a 95\% predictive interval for the total traffic on a new residential street 
with a bike lane. 
\end{enumerate}
\newpage

\{} 
\begin{table}
\centering
\begin{tabular}{ccc}
       & Estimated      & Standard error       \\
       & treatment      & of effect            \\
School & effect, $y_j$  & estimate, $\sigma_j$ \\
\hline
A & 28 & 15 \\
B &  8 & 10 \\
C & -3 & 16 \\
D &  7 & 11 \\
E & -1 &  9 \\
F &  1 & 11 \\  
G & 18 & 10 \\
H & 12 & 18 
\end{tabular}
\caption{{\em Observed effects of special preparation on SAT-V scores in eight 
randomized experiments.  Estimates are based on separate analyses for the eight 
experiments.  From Rubin (1981).}}
\end{table}


\begin{table}
\centering
\begin{tabular}{lll}
Type of   & Bike &  Counts of bicycles/other vehicles \\
street     & route?   &     \\
\hline
Residential & yes & 16/58, 9/90, 10/48, 13/57, 19/103,           \\
            &     & 20/57, 18/86, 17/112, 35/273, 55/64          \\
Residential & no  & 12/113, 1/18, 2/14, 4/44, 9/208,             \\
            &     & 7/67, 9/29, 8/154                            \\
Fairly busy & yes & 8/29, 35/415, 31/425, 19/42, 38/180,         \\
            &     & 47/675, 44/620, 44/437, 29/47, 18/462        \\
Fairly busy & no  & 10/557, 43/1258, 5/499, 14/601, 58/1163,     \\
            &     & 15/700, 0/90, 47/1093, 51/1459, 32/1086      \\
Busy        & yes & 60/1545, 51/1499, 58/1598, 59/503, 53/407,   \\
            &     & 68/1494, 68/1558, 60/1706, 71/476, 63/752    \\
Busy        & no  & 8/1248, 9/1246, 6/1596, 9/1765, 19/1290,     \\
            &     & 61/2498, 31/2346, 75/3101, 14/1918, 25/2318
\end{tabular}
\caption{{\em Counts of bicycles and other vehicles in one hour in each of 10 city blocks
in each of six categories.  (The data for two of the residential blocks were lost.)
For example, the first block had 16 bicycles and 58 other vehicles, the second had
9 bicycles and 90 other vehicles, and so on.  Streets were classified as 'residential,'
 'fairly busy,' or 'busy' before the data were gathered.}}
\end{table}

\end{document}

